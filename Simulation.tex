\documentclass[a4paper,11pt,twoside]{article}
\usepackage[german]{babel}
\usepackage[utf8]{inputenc}
\usepackage[T1]{fontenc}
\usepackage[svgnames]{xcolor}
\usepackage{amsmath, amsfonts, amssymb, graphicx, flafter, multirow, fancyhdr}
\usepackage[pdftex, colorlinks=true,linkcolor=DarkBlue, urlcolor=black, citecolor=DarkGreen]{hyperref}
\pagestyle{fancy}
\fancyhf{}
\fancyhead[L]{{\small FOPRA - Physics Simulation with Geant4}}
%%\fancyhead[C]{{\small Lorenz Schlechter, \\Thomas Kraetzschmar}}
\fancyhead[R]{{\small\date{\today}}}
\fancyfoot[C]{\thepage}
\pagestyle{fancy}
\renewcommand{\topfraction}{0.9}
\renewcommand{\bottomfraction}{0.6}
\renewcommand{\textfraction}{0.1}
\setcounter{topnumber}{3}
% Title Page
\title{%
{\Huge Physics Simulation with Geant4}\\[0.5\baselineskip]
{\normalsize Gruppe 51}
}
\author{%
Thomas Kraetzschmar
\and Lorenz Schlechter
\and Maximilian Ziegler
}
\date{\today}
%#############################################################################
\begin{document}
\pagestyle{fancy}
\pagenumbering{roman}
\maketitle
\clearpage
%\cleardoublepage
\tableofcontents
\clearpage
\pagestyle{fancy}
\pagenumbering{arabic}
%*************************************************************************************
\section{Introduction}

\section{Messung}
\subsection{Versuchsaufbau}
Um später die Genauigkeit der Simulation bestimmen zu können wurde zuerst der Simulierte Aufbau tatsächlich umgesetzt. Hierbei wurde eine Na-22 Quelle verqwendet. Diese zerfällt in einem $\beta^+$-Zerfall zu einem angeregten $^{22}Ne$, welches durch Emission eines Photons in den Grundzustand übergeht. Dieses Photon hat eine Energie von 1275 keV. Darüber hinaus kann das bei  dem $\beta^+$-Zerfall entstandene Positron mit einem Hüllenelektron annihilieren, wobei zwei Photonen der Energie 511 keV entsehen.
Diese Photonen werden mittels Caesiumiodid Szintillationskristallen detektiert. \\\\Die Detektoren werden dabei auf vier verschiedene weisen platziert:
\begin{enumerate}
\item Die Quelle befindet sich in 3,5cm Höhe über dem Tisch, eine kleine Detektorbox befindet sich dabei in einem Abstand von 3cm, wobei der Detektorkristall sich 1cm tief in der Box befindet, der Gesamtabstand beträgt damit 4cm.
\item Gleicher Aufbau nur mit einem Abstand von 5cm bzw. einem Gesmatabstand von 6 cm.
\item Die Quelle befindet sich in 3,5cm Höhe und 3cm vor dem großen Detektor.
\item Die Quelle befindet sich in 3,5cm Höhe, auf der einen Seite befindet sich ein kleiner Detektor im Abstand von 5 cm, auf der anderen Seite ein kleiner Detektor über dem großen Detektor im gleichen Abstand.
\end{enumerate}
Die Messungen dauerten bei den ersten beidne Messungen 7 Minuten, bei der dritten 9 Minuten und bei dem vierten Aufbau 25 Minuten.
\subsection{Aktivität}
Die Aktivität der Probe wurde am 1.11.2013 zu $A_0=213 kBq$ bestimmt. Der Versuch wurde am 23.10.2014 durchgeführt, was einer Zeitdifferenz t von 357 Tagen entspricht. Die Halbwertszeit $T_{1/2}$ beträgt 2,6 Jahre. Damit beträgt die Aktivität zum Versuchszeitpunkt:

\begin{equation}
A=A_0*0,5^{t\over T_{1/2}}=164,11 kBq
\end{equation}

\subsection{Auswertung}
Die Messungen liefern die Anzahl der Ereignisse pro Kanal. Physikalisch interessant ist allerdings die Energie. Da die Energie der Peaks bekannt ist kann man daraus den Kanälen Energien zuordnen. Bei der ersten Messung liegen die Peaks bei den Kanälen $4316,6\pm0,4$ und $10572\pm2$. Diese Entsprechen den Energien 511keV und 1275keV. Daraus ergibt sich die Zuordnung:
\begin{equation}
E=0,1221422862*K-16,2882494
\end{equation}
Wobei K die Kanalnummer ist.
Um die Auflösung der Detektoren zu bestimmen werden die Peaks über eine Gaußkurve angenähert. Die Auflösung wird als die Halbwertsbreite definiert, die bei einer Normalverteilung bei 1,35$\sigma$ liegt. Bei der ersten Messung ergibt sich bei einem Fit von Kanal 4000 bis 4700 ein $\sigma$ von $129,6\pm0,4$. Ein Fit von Kanal 10100 bis 11000 liefert ein Sigma von $178\pm2$



%\bibliographystyle{alpha}
%\bibliographystyle{apalike}
%\bibliographystyle{plain}
%\bibliography{Literaturverzeichniss-Brue}
%\appendix
%\listoffigures
%\listoftables
%
\end{document} 